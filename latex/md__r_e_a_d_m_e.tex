\subsection*{Setting up the project}

To develop the Stand Alone Braille Tutor, follow these steps.

\subsubsection*{Required Software and Hardware}


\begin{DoxyItemize}
\item Windows X\-P or later
\item A\-V\-R Studio 4. Later versions may also work, but this guide assumes the use of A\-V\-R Studio 4.
\item An A\-V\-R J\-T\-A\-G\-I\-C\-E mk\-I\-I programmer.
\item Two U\-S\-B type A to U\-S\-B type B cables.
\item Headphones or a speaker system with analog input.
\item Git or Github for Windows.
\end{DoxyItemize}

\subsubsection*{Downloading the project from Git\-Hub}


\begin{DoxyEnumerate}
\item Download and install Github for Windows. This can be obtained from windows.\-github.\-com.
\item Obtain access to the repository at \href{https://github.com/CMU-15-239/SABTSoftware}{\tt https\-://github.\-com/\-C\-M\-U-\/15-\/239/\-S\-A\-B\-T\-Software}.
\item Click the \char`\"{}\-Clone in Windows\char`\"{} button and choose a location for the repository. This guide will assume the repository is copied to Documents.
\end{DoxyEnumerate}

\subsubsection*{Compiling the software}

See the programming guide -\/ S\-A\-B\-T\-\_\-programming\-\_\-guide.\-pdf.

\subsubsection*{Loading the software binaries onto the S\-A\-B\-T hardware}

See the programming guide -\/ S\-A\-B\-T\-\_\-programming\-\_\-guide.\-pdf.

\subsection*{Components}

There are four different boards that compose the S\-A\-B\-T -\/ the Main Control Unit and one of three User Interface boards. The Main Control Unit is used in conjunction with one of the other three boards to make an interface suitable for the user's skill level. \subsubsection*{Main Control Unit (M\-C\-U)}

This folder contains all of the code for the control unit which does all of the processing for the boards and contains the modes. Open S\-A\-B\-T\-\_\-\-Main\-Unit.\-aps in A\-V\-R Studio 4. The M\-C\-U handles the following\-:
\begin{DoxyItemize}
\item Provides connection (via U\-S\-B cable) to a P\-C for debugging purposes / mode selection
\item Provides battery power / power from U\-S\-B
\item Has an S\-D card slot for sound files
\item Plays sound files
\item Handles different Modes
\end{DoxyItemize}

\subsubsection*{Primary User Interface Board (Primary U\-I)}

This folder contains all of the code for the first user interface board -\/ the board with just six large braille dots. Open S\-A\-B\-T\-\_\-\-Primary.\-aps in A\-V\-R Studio 4.
\begin{DoxyItemize}
\item Consists of six large braille dots, essentially one large cell
\item Has volume control buttons, mode control buttons, enter buttons
\item Transmits signals to M\-C\-U when a button is pressed for processing by the M\-C\-U
\end{DoxyItemize}

\subsubsection*{Intermediate User Interface Board (Intermediate U\-I)}

No work has been done yet in this folder, but it should contain the file S\-A\-B\-T\-\_\-\-Intermediate.\-aps, and handle the intermediate user interface board.
\begin{DoxyItemize}
\item Consists of three large braille cells and two rows of actual sized braille cells (16 cells per row)
\end{DoxyItemize}

\subsubsection*{Advanced User Interface Board (Advanced U\-I)}

No work has been done yet in this folder, but it should contain the file S\-A\-B\-T\-\_\-\-Advanced.\-aps, and handle the advanced user interface board.
\begin{DoxyItemize}
\item Consists of six rows of braille cells (16 cells per row)
\end{DoxyItemize}

\subsection*{Current issues and pitfalls}


\begin{DoxyItemize}
\item We need a better way to read in the dictionary more quickly. It currently takes 10-\/15 seconds which is a distracting amount of silence.
\item Pressing a volume button while an mp3 file is playing breaks the system.
\item We are currently receiving a warning -\/ fixing the warning causes the code to not function, so there must be another way to resolve the warning such that the code can still run. This warning is\-: {\ttfamily ../\-F\-A\-T32.c\-: In function 'read\-\_\-and\-\_\-retrieve\-\_\-file\-\_\-contents'\-:} {\ttfamily ../\-F\-A\-T32.c\-:360\-: warning\-: 'num\-\_\-bytes\-\_\-read' may be used uninitialized in this function}
\end{DoxyItemize}

\subsection*{Opportunities for further development}


\begin{DoxyItemize}
\item The Intermediate and Advanced user interface boards need to be developed in their entirety.
\item Users have expressed interest in two different voices -\/ one for input and one for output. This would involve recording one voice (preferably masculine) for all of the prompts and instructions given by the device, and a second voice (ideally feminine) to echo user input. For example, The instructions for Letter Practice and the letter prompts should be in the first voice, while the voice echoing the dot input should be the second voice.
\item Users have expressed interest in a mode that serves as a reference for how to write each letter. In other words, this mode would be Letter Practice without any user input -\/ it would just read out how to form each letter. This could potentially be done by using the mode select buttons to scroll between letters instead of reading out eery letter.
\item From user testing, we found that some users entered dots very quickly. It would be very helpful to have either some mechanism to skip M\-P3s (if a user already knows the prompt) or some way to adjust the speed of the M\-P3 playback.
\item The Animal Game could be expanded to include more animal sounds, and to include a variety of animals such that all of the letters are tested at some point.
\item In the Animal Game, users wanted to be able to skip animals.
\item A couple of users experessed interested in a \char`\"{}\-Household Sounds\char`\"{} game -\/ much like the animal game, but the sounds are sounds of everyday life.
\item Sentence writing practice -\/ This could be a feature especially for the Intermediate and Advanced boards which have slate rows.
\item One player hangman could be set up to use a random word from the entire dictionary as opposed to the fixed current list we provide.
\end{DoxyItemize}

\subsection*{Doxygen Documentation}

Documentation at \href{http://cmu-15-239.github.com/SABTSoftware/}{\tt http\-://cmu-\/15-\/239.\-github.\-com/\-S\-A\-B\-T\-Software/} 